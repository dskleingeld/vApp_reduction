% Chapter X

\chapter{Theory} % Chapter title
\label{ch:theory} % For referencing the chapter elsewhere, use \autoref{ch:name} 
%intro
Here we discuss what an psf is, how it helps us describe an optical system and how they natuarally apear from basic optics. Then we look at the different types of disks that have been observed and what contrasts we expect.
%------------------------------------------------
\section{PSF}
\label{sec:PSF}

We can describe what happens to light going through an optical system with its point spread function (PSF). It describes the light intensity on the focal plane (where the sensor is located) as a function of x and y when a single light ray is imaged on the center of the focal plane. 

We derive a way to find the PSF from the Hygens-Fresnel Principle. It states any part of a wave can be described as a front of inifinite point sources interfering with one another. An optical element can change these arrangement of theses sources, for example an aperture allows only a small area to be filled with these point sources as illustrated in \autoref{fig:hecht}. 

\begin{figure}[h]
    \caption{Fraunhofer diffraction from an arbitary aperture, r and R large compared to the size of the hole. Extracted from Optics 5th edition, by \cite{hecht}}
    \centering
    \includegraphics[width=\textwidth]{gfx/hecht2}
    \label{fig:hecht}
\end{figure}

We find the electric field at a point P at distance R by summing up the fields of these infinite point sources taking into account the different distances to R. Writing the infinite sum as an integral we get \autoref{eq:hecht10.41} for the electric field at a point P some distance R from an apperture.

\begin{subequations}
    \begin{align}
        E &= \frac{\varepsilon_A e^{i(wt-kR)}}{R} \iint_{Aperture} e^{ik(Yy+Zz)/R} dS
    \end{align}
    \label{eq:hecht10.41}
\end{subequations}

Here capital Y, Z describe the position in the imaging plane in which P lies as seen in \autoref{fig:hecht}. Small letters y and z are the position in the apperture plane. The integral is over the apperture, only integrating over the transparent parts. 

To account for changes in phase and not only magnitude of the field caused by Optical instruments we including an aperture function instead of just integrating over the shape of the apperture. This results in:

\begin{subequations}
  \begin{align}
    \mathscr{A}(y,z) &= \mathscr{A}_0 (y,z)e^{i\upphi(y,z)}
  \end{align}
\end{subequations}

Here the amplitude of the apperture function comes from $\mathscr{A}_0$ and the phase from $e^{i\upphi(y,z)}$.

\begin{subequations}
  \begin{align}
    E(Y,Z) &= \iint_{-\infty}^{\infty} \mathscr{A}(y,z) e^{ik(Yy+Zz)/R} dydz
  \end{align}
\end{subequations}

The expression for the E field at the point P (\autoref{eq:hecht10.41}) rewritten to make use of the apperture function.

We can rewrite this to get rid of the dependence on the distance by subsututing Ky = kY/R and Kz = kZ/R for Y and Z. This gives the final form: 

\begin{subequations}
%  \label{eq:Master} % NO! impossible here
  \begin{align}
    E(K_Y,K_Z) &= \iint_{-\infty}^{\infty} \mathscr{A}(y,z) e^{ik(K_Yy+K_Zz)/R} dydz
  \end{align}
\end{subequations}

This is the 2 dimensional Foerier transformation of the aperture function. Thus "the field distribution in the fraunhofer diffraction pattern is the Fourier transform of the field distrubution across the aperture (e.i., the aperture function" \cite{hecht}). 

For the psf we are interested in the intesity which is not the electric field $E$ but ${|E|}^2$. This means we can calculate the PSF of the vAPP by foerier transforming its apperture function and squaring the result. Note that the amplitude in the apperture function does not only have to depend on the shape of the apperture as there might be partially transparant material. The apperture function does not even have to be an apperture.

With the PSF we know how a single point source would look when imaged by an optical system. If we assume the system is linear system (****isnt it always*****?) we can use the superposition principle to image extended sources by convolving the psf with the extended source. 

\subsection{Atmorphere}
% how does the atmospere and AO influence the instrument psf

The telescope or choronograph are the only optical system at play. There are many differently moving layers of air between the telescope and space. These work as independent optical systems that change in time. Each layer moves in a different direction at a different speed as winds are different at various altitudes. This changes the phase of the light. The complete psf changes all the time. The changes are smaller at smaller timescales. 

\subsection{Adaptive Opics and spechles}
Since the 1990 adaptive optics (AO) are used. These change shape to undo the phase change of the atmosphere. However as these effects are unpredictable they always lack behind slightly. Further more they have errors themself, the phase is never completly corrected. Both these effects cause small distortions in the final image, these are known as spechles. Thus even with AO the total psf for the atmosphere, the AO and the instrument will keep changing in time, however the magnitude of the change is severly reduced. Adaptive optics don not correct the entire field, generally each system has a radius of correction in which the AO works.

%---------------------------------------------------------------------------------------



-

\section{vAPP}

As mentioned in the introduction the vector apodising phase plate is not a normal coronograph. The apodising phase plate (APP) is a coronograph that changes amplitude of phase in the pupil plane to create destructive interference in an erea of the psf. This creates a very dark area on the psf. Dim object imaged there can be resolved if the contrast between them and the star is smaller then the contrast between center peak of the psf and the dark zone. Because the coronograph works in the pupil plane it is insensitive to the effects of spechles, further more unresolved stars do not limit how close to a star the chronograph can function. %TODO is this last sentence correct, paper names tip tilt erros and unresolved stars? 

The APP does this by introduces differences in the optical path length thereby changing the phase. These changes are designed to create the mentiond dark zone in the shape of a $180^\circ$ half circle with a radius of 2 to 9 $\lambda/D$ \cite{vAPP_vs_APP}. To do so opical the path length differences need to be different throughout the pupil, the design can be seen as a hightmap of path differences. The APP is manufactured by directly printing that heightmap using liquid-crystal technology. 

The vAPP is an upgrade to the APP that applies a the path length 
%FIXME vapp creates 2, left and right handed, mirrord copies that are then split based on handedness?
%TODO write text
%---------------------------------------------------------------------------------------






\section{Disks} % Chapter title
\label{sec:disks} % For referencing the chapter elsewhere, use \autoref{ch:name} 

A newly formed star is enveloped in a disk of gass and dust, a protoplanatary disk.

--something about the components/dust/gass ref{williams}

From it planets can form. The process of planat formation is not yet understoot. Direct imaging of newly formed planets is hard. Objects inside a disk disturb the shape and create features in the disk. By observing the features of a disk we can learn more about planet formation.

\subsection{Catogorisation}

Roughly speaking disks seem to be ring or spiral shaped with some forms in between. \cite{garufi} classifies them into 6 catogories in \autoref{fig:sketch_garufi}. 

\begin{figure}[h]
    \caption{Sketch summarizing the different classifications of protoplenatary disks proposed by /ref}
    \centering
    \includegraphics[width=\textwidth]{gfx/catogories}
    \label{fig:sketch_garufi}
\end{figure}

They then conclude that: 
-faint disks are young
-spiral disks are almost starting theire main sequence
-ring disk have no outer stellar companion (??)  

%------------------------------------------------

\subsection{Challenging to Obeserve}

It is challanging to observe a disk since its brightness is low compared to the star. To get an upper limit on the brightness of a disk assume all star light that hits the disk is reflected towards us. The light from a star drops quadratically as it gets farther away. Thus we will never have a disk brightness exeeding $1/R^2$. A disk at $1$ AU from an run like star will have a brightness \num{2.1e-5} of the star. At $5$ AU this drops to \num{8.6e-7}. 

%TODO something about contrast

%Currently there is only one planet that has been detected with direct imaging within a 1 EU radius, however it orbits an ultracool dwarf /ref{http://exoplanet.eu/catalog/de0823-49_b/}.

%----------------------------------------------------------------------------------------
