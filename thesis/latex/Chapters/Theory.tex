% Chapter X

\makeatletter
\def\user@resume{resume}
\def\user@intermezzo{intermezzo}
%
\newcounter{previousequation}
\newcounter{lastsubequation}
\newcounter{savedparentequation}
\setcounter{savedparentequation}{1}
% 
\renewenvironment{subequations}[1][]{%
      \def\user@decides{#1}%
      \setcounter{previousequation}{\value{equation}}%
      \ifx\user@decides\user@resume 
           \setcounter{equation}{\value{savedparentequation}}%
      \else  
      \ifx\user@decides\user@intermezzo
           \refstepcounter{equation}%
      \else
           \setcounter{lastsubequation}{0}%
           \refstepcounter{equation}%
      \fi\fi
      \protected@edef\theHparentequation{%
          \@ifundefined {theHequation}\theequation \theHequation}%
      \protected@edef\theparentequation{\theequation}%
      \setcounter{parentequation}{\value{equation}}%
      \ifx\user@decides\user@resume 
           \setcounter{equation}{\value{lastsubequation}}%
         \else
           \setcounter{equation}{0}%
      \fi
      \def\theequation  {\theparentequation  \alph{equation}}%
      \def\theHequation {\theHparentequation \alph{equation}}%
      \ignorespaces
}{%
%  \arabic{equation};\arabic{savedparentequation};\arabic{lastsubequation}
  \ifx\user@decides\user@resume
       \setcounter{lastsubequation}{\value{equation}}%
       \setcounter{equation}{\value{previousequation}}%
  \else
  \ifx\user@decides\user@intermezzo
       \setcounter{equation}{\value{parentequation}}%
  \else
       \setcounter{lastsubequation}{\value{equation}}%
       \setcounter{savedparentequation}{\value{parentequation}}%
       \setcounter{equation}{\value{parentequation}}%
  \fi\fi
%  \arabic{equation};\arabic{savedparentequation};\arabic{lastsubequation}
  \ignorespacesafterend
}
\makeatother

\chapter{Theory} % Chapter title
\label{ch:theory} % For referencing the chapter elsewhere, use \autoref{ch:name} 
%intro
Here we discuss what an \ac{PSF} is, how it helps us describe an optical system and how they naturally appear from basic optics. Then we look at the different types of disks that have been observed and what contrasts we expect.
%------------------------------------------------
\section{Point spread function}
\label{sec:PSF}

We can describe what happens to light going through an optical system with its \ac{PSF}. It describes the light intensity on the focal-plane (where the science detector is) as a function of x and y when a single point source is imaged on the center of the focal-plane. 

Light is an propagating electromagnetic wave. We derive how to find the \ac{PSF} from the Huygens-Fresnel Principle. It states that any part of a wave can be described as a front of infinitely many point sources interfering with one another. An optical element can change these arrangement of theses sources, for example an aperture allows only a small area to be filled with these point sources as illustrated in \autoref{fig:hecht}. 

\begin{figure}[h]
    \caption{Fraunhofer diffraction from an arbitrary aperture, r and R large compared to the size of the hole. Extracted from Optics 5th edition, by \cite{hecht}.}
    \centering
    \includegraphics[width=\textwidth]{gfx/hecht2}
    \label{fig:hecht}
\end{figure}

We find the electric field at a point P at distance R by summing up the fields of these infinite point sources taking into account the different distances to R. Writing the infinite sum as an integral we get \autoref{eq:hecht10.41} for the electric field at a point P some distance R from an aperture.

\begin{subequations}
    \begin{align}
        E &= \frac{\varepsilon_A e^{i(wt-kR)}}{R} \iint_{\text{Aperture}} e^{ik(Yy+Zz)/R} dydz%
        \label{eq:hecht10.41}
    \end{align}
\end{subequations}

Here $Y$, $Z$ describe the position in the imaging plane in which $P$ lies as seen in \autoref{fig:hecht}. Small letters $y$ and $z$ are the position in the aperture plane. The integral is over the aperture, only integrating over the transparent parts. 

Here we integrate only over the aperture and assume all light is transmitted within and no light outside of it. To allow for changes in phase and magnitude caused by for example dirty glass, or a complicated optical instrument don not integrate over only an aperture but use an aperture function and integrate over all space.

\begin{align}
\mathscr{A}(y,z) &= \mathscr{A}_0 (y,z)e^{i\upphi(y,z)}\label{eq:helper}
\end{align}

Here the time changing electromagnetic wave $\varepsilon$ is averaged into a constant field of magnitude one. Then the transmission through the aperture changes the magnitude ($\mathscr{A}_0$) and the phase ($\upphi(y,z)$).

\begin{subequations}[resume]
  \begin{align}
    E(Y,Z) &= \iint_{-\infty}^{\infty} \mathscr{A}(y,z) e^{ik(Yy+Zz)/R} dydz%
    \label{eq:hecht:2}
  \end{align}
\end{subequations}

The expression for the E field at the point P (\autoref{eq:hecht10.41}) rewritten to make use of the aperture function.

We can rewrite this to get rid of the dependence on the distance by substituting $K_y = kY/R$ and $K_z = kZ/R$ for $Y$ and $Z$. This gives the final form: 

\begin{subequations}[resume]
%  \label{eq:Master} % NO! impossible here
  \begin{align}
    E(K_Y,K_Z) &= \iint_{-\infty}^{\infty} \mathscr{A}(y,z) e^{i(K_Yy+K_Zz)} dydz%
    \label{eq:hecht:3}
  \end{align}
\end{subequations}

This is the 2 dimensional Fourier transformation of the aperture function. Thus \textquote{the field distribution in the Fraunhofer diffraction pattern is the Fourier transform of the field distribution across the aperture (i.e. the aperture function)} \cite{hecht}. Here the Fraunhofer diffraction pattern is the instruments focal-plane.

For the \ac{PSF} we are interested in the intensity which is not the electric field $E$ but ${|E|}^2$. This means we can calculate the \ac{PSF} of an instrument by Fourier transforming its (complex) aperture function and squaring the result. Note that the amplitude in the aperture function does not only have to depend on the shape of the aperture as there might be partially transparent material. We can use it to simulate any optical instrument such as the \acf*{vAPP}.

With the \ac{PSF} we know how a single point source would look when imaged by an optical system. If we assume the system is linear system we can use the superposition principle to image extended sources by convolving the \ac{PSF} with the extended source. 
%TODO above paragraph isnt correct see feedback, fix it!.

\subsection{Atmosphere}
% how does the atmospere and AO influence the instrument psf

The telescope or coronagraph are not the only optical system at play. There are many differently moving layers of air between the telescope and space. These work as independent optical systems that change in time. The varying temperature and humidity between and within layers changes the air's refractive index resulting in phase changes. Each layer moves in a different direction at a different speed as winds differ between altitudes. Combined this causes the complete \ac{PSF} to change continuously. The changes are smaller at smaller timescales as the layers move less far and temperature and humidity change less. 

\subsection{Adaptive Optics}
\label{sec:ao}
Since the 1990‘s \ac{AO} are used. They utilize an adaptive mirror that can change its shape to undo the phase changing of the atmosphere, see \autoref{fig:ao}. To do this they first need to measure the errors in the wavefront. However as these measurement are not instant the shape of the mirror can never match the current wavefront. There are also some errors in positioning the mirror will always match the required shape. Both these effects cause small distortions in the final image. Thus even with \ac{AO} the total \ac{PSF} for the atmosphere, the \ac{AO}, the telescope and the instrument will keep changing in time, however the magnitude of the change is severely reduced. Adaptive optics do not correct the entire field. Each system has a control radius around the star in which the \ac{AO} can reduce the seeing.

\begin{figure}[H]
    \caption{Sketch showing the basic principle of an \ac{AO} system, adapted from UC Berkeley science review: Into focus, Credit: Keith Cheveralls.}
    \centering
    \includegraphics[width=\textwidth]{gfx/AO2}
    \label{fig:ao}
\end{figure}

%---------------------------------------------------------------------------------------




\section{Vector Apodizing Phase Plate}
\label{sec:vapp}

As mentioned in the introduction, the \ac{vAPP} is not a conventional coronagraph that blocks light. It is a coronagraph that changes the phase in the pupil-plane to create destructive interference in an area of the \ac{PSF}. This creates a dark zone in the \ac{PSF}. Dim objects can be imaged if the contrast between them and the star is smaller than the contrast between center peak of the \ac{PSF} and the noise in the dark zone. The contrast can be higher if post processing techniques such as \ac*{ADI} are used (see \autoref{sec:adi}).

As the coronagraph works in the pupil-plane it is insensitive to the effects of tip tilt errors, which are caused by vibrations in the telescope or imperfectly corrected atmospheric tip tilt effects. Further more unresolved stars do not limit how close to a star the coronagraph can function.

The \ac{vAPP} iterates on the \ac{APP}, it creates its destructive interference by introducing differences in the optical path length thereby changing the phase. These changes are designed to create the mentioned dark zone in the shape of a $180^\circ$ half circle with a radius of 2 to 9 $\lambda/D$ \cite{vapp_doelman}. To achieve these phase changes the path length differences need to change throughout the pupil. The design is stored as a heightmap of the required path differences. It is then manufactured by diamond turning glass. 

Instead of modifying the optical path length, the \ac{vAPP} introduces the phase changes using the geometrical properties of a liquid crystal plate. The two circular polarization receive opposite phase and create dark holes at opposite sides of the \ac{PSF} \cite{vapp_snik} \cite{vapp_otten} \cite{vapp_doelman}. By then splitting the right and left handed light and imaging both separated we get not $180^\circ$ of dark zone but $360^\circ$ solving one of the major problems of the \ac{APP}. Here we refer to the small \acp{PSF} within the full \ac{PSF} as coronagraphic \acp{PSF}, see \autoref{fig:into_vapp_annotated} in \autoref{ch:intro}.
%---------------------------------------------------------------------------------------






\section{Disks} % Chapter title
\label{sec:disks} % For referencing the chapter elsewhere, use \autoref{ch:name} 

With the formation of a star it is inevitable a disk will form due to to the conservation of angular momentum. While initially material will orbit in various directions and planes these will slowly cancel out due to collisions and attraction leaving the average plane and direction as the final orientation of a disk around the star. It seems likely these disk will allow the formation of planets due to the high detection rate of exoplanets \cite{williams}.

The formation of disks goes through 3 main stages\cite{williams}. Only the last stage allows direct imaging in the near-IR and optical bands though all are observable in Radio:

\begin{enumerate}
  \item Right after the molecular core collapses, most mass is still in a cloud surrounding the disk and the star. Due to that it is not possible to see the disk in the near-IR  or optical regime.
  \item Most mass has moved into the star, the disk is partially obstructed by outflow of mass from the star. The enveloping mass is about the same as the disks.
  \item Central star becomes visible, the enveloping dust has cleared or accreted onto the disk and the disk only contains a few percent of the total mass of the system. It now is a protoplanetary disk.
\end{enumerate}

During the first two stages the disk is obstructed by dust. These stages take a relatively short time. However at the third stage we should be able to study the disk. During this stage planets can form. Their formation process is uncertain. 

\subsection{Categorization and features}

It is suspected the disks features are linked to the formation of planets. Rings in a disk could indicate a planet embedded in the disk \cite{rings} however those planets can currently not be detected. There is also evidence that spirals in disks can be caused by planets in the disk \cite{garufi}. This means the morphology of disks can provide valuable insight in how planets form.

Disks seem to be ring or spiral shaped with some forms in between. They can be classified into 6 categories \cite{garufi} see \autoref{fig:sketch_garufi}. 

\begin{figure}[H]
    \caption{Sketch summarizing the different classifications of protoplanetary disks proposed in \cite{garufi}.}
    \centering
    \includegraphics[width=\textwidth]{gfx/catogories}
    \label{fig:sketch_garufi}
\end{figure}

With this categorization which is still topic of much debate they \cite{garufi} further conclude:
\begin{itemize}
    \item faint disks are young
    \item spiral disks are around stars that almost start their main sequence
    \item ring disk have no outer stellar companion %TODO explain
\end{itemize}

Furthermore they \cite{garufi} conclude that in small disks of 10 to 20 Au in size and young faint disks structures remain undetected. Observing disks using the \acp{vAPP} which aims to allow imaging at few $\lambda/D$ could allow us to detect these structures.
%----------------------------------------------------------------------------------------

\section{Angular differential Imaging}
\label{sec:adi}
\ac{ADI} \cite{marois_2005} is a technique to remove stationary and slowly changing PSF structures from an image. It is a proven method for the detection of companions with classic \acp{PSF} such as the one in \autoref{fig:classic_psf}. %TODO something about vAPP?

% theory adi 
% TODO [trivial] field of view as \ac{FOV} ?
The basis of \ac{ADI} is to let the field of view rotate with respect to the coronagraph and detector during an observation. By then taking multiple short images anything off center will be rotated a bit in each image. If the mean of all other images is subtracted from an image we lose most of the stellar \ac{PSF}. Once done for every image, rotating all the results so the field of view is aligned the median can be taken which gives the final image.

% vAPP adi theory
To use \ac{ADI} with the \ac{vAPP} we make sure the entire sequence fits in one of the two dark holes. This makes post processing easier as we do not need to worry about merging the two dark holes.
