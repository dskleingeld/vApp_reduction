% Chapter X

\chapter{Data generation} % Chapter title
\label{ch:data_gen} % For referencing the chapter elsewhere, use \autoref{ch:name} 

There are no obeservations of disks with the \ac{vAPP} here we describe how we generate observations using a disk model and \acp{PSF}. This presents a number of advantages though the main reason is the lack of disk observations using the \ac{vAPP}. Generating the data from a model allows us to vary paramaters as we wish. We can try a reduction method on a simple slightly inclined disk or a near face on disk with many rings. By reducing images with varying disk morphology we can map how a reduction method performs for each disk morphology. Most importantly we can clearly separate artifacts created during data reduction from disk features when we know what the disk looks like.

%------------------------------------------------




\section{Disk model}

We use a 2-dimensional disk model based on the work by \cite{Pieter_Okko}. The model has 4 basic parameters: 

\begin{description}
\item[Inclination] The angle the disk is tilted towards the observer. A 0 degrees inclination gives a face on disk and 90 degrees a horizontal line being an edge on disk. As illustrated in \autoref{fig:inclination}
\item[Position angle] After inclination the disk can be rotated around the line of sight from the observer, rotation to the left is positive. See \autoref{fig:pos_angle}
\item[Inner and Outer radius] Many disks start and stop at some radius from star. The inner and outer radius are relative to the field size that defaults to 10. An Inner radius of 2 gives a hole in the disk with a diameter 20\% of the image width. 
\end{description}

\begin{figure}[h!]
  \centering
  \begin{subfigure}[b]{0.4\textwidth}
    \includegraphics[width=\textwidth]{gfx/inclination}
    \caption{Inclination angle from: \cite{Pieter_Okko}}
    \label{fig:pos_angle}

  \end{subfigure}
  \begin{subfigure}[b]{0.4\textwidth}
    \includegraphics[width=\textwidth]{gfx/pos_angle}
    \caption{Position angle, from: \cite{Pieter_Okko}}
    \label{fig:inclination}
  \end{subfigure}
  \label{fig:1}
\end{figure}

The disk is modeled as optically thick and does not emit light on its own. At a certain radii between the inner and outer radii the disks brightness is given by:

\begin{equation}
B(r) = B_{star} \cdot {\left(\frac{r}{R_{star}}\right)}^2
\end{equation}

We evaluate the model onto an image of 200 by 200 pixels. This speeds up calculations and is similar to the resolution of observed \ac{vAPP} datasets. The model behaves well for most disks however on this resolution features that are small in the observers plane are pixilated as we see in \autoref{fig:disk} below. This poses no problem as we do not expect to resolve such features.

\begin{figure}[h!]
  \begin{subfigure}[b]{0.5\textwidth}
    \includegraphics[width=1.2\linewidth]{gfx/disk_45_60}
    \caption{Model output for an inclination of $60\deg$, an inner radius of $2$ and an outer radius of $5$.}
  \end{subfigure}
  \begin{subfigure}[b]{0.5\textwidth}
    \includegraphics[width=1.2\linewidth]{gfx/disk_45_10_22_23}
    \caption{Model output for an inclination of $80\deg$, an inner radius of $2.2$ and an outer radius of $2.3$.}
  \end{subfigure}
  \caption{Two disks created by the model with different size and inclination both with a position angle of 45 degrees.}
  \label{fig:disk}
\end{figure}

%------------------------------------------------




\section{Atmospheric Distortions}

To simulate an observation we do not use a single given psf. Since during an observation the combined psf of the atmosphere and instrument changes. However a complete simulation of these effects is outside the scope of the thesis. Instead try and approach a similar morphology to a on sky psf, with similar changes in time. We try two different methods, adding distortions on top of an existing \ac{vAPP} \ac{PSF} and adding an atmosphere simulation to the PSF generation. Finally we compare the result of each method to on sky observations.

\subsection{Fourier method}
First we tried modifying a give single \ac{vAPP} psf to get a set of disturbed psfs. To achieve this a pattern is added to the foerier transform of the psf before transforming it back from foerier space, see the equation below. The pattern is then shifted for every timestap.

\begin{equation}
I_{final} = \mathscr{F}_{2d}^{-1} \Big( intensity \cdot \mathscr{F}_{2d}(psf) \cdot pattern + (1-intensity) \cdot \mathscr{F}_{2d}(psf) \Big)
\end{equation}

The best results where achieved using a grid of blurred circles as pattern, see \autoref{fig:patterns}. Note the distortions to the \ac{PSF} are around the center of its peaks but not randomly spread. As Clearly this will not do.

\begin{figure}[h!]
      \begin{subfigure}[t]{0.5\textwidth}
        \includegraphics[width=1.2\linewidth]{gfx/0_old_psf_diff1}
        \caption{Difference between two \acp{PSF}  created using this method, between \acp{PSF} the pattern has shifted 10 pixels.}
      \end{subfigure}% this comment sign needs to be here for the images to be on the same line
      \begin{subfigure}[t]{0.5\textwidth}
        \includegraphics[width=1.2\linewidth]{gfx/0_old_psf_diff2}
        \caption{Difference between two \acp{PSF}  created using this method, between \acp{PSF} the pattern has shifted 5 pixels.}
      \end{subfigure}
      
      \begin{subfigure}[]{0.5\textwidth}
        \includegraphics[width=1.2\linewidth]{gfx/0_pattern}
        \caption{Pattern that is shifted and applied to the foerier space of a \ac{PSF}.}
      \end{subfigure}
      
  \caption{The Pattern used and the differences between \acp{PSF} distorted with it using the method described above.}
  \label{fig:patterns}
\end{figure}

\subsection{Using HCIPy}
\label{sec:hcipy}
Then we tried a simulation of the psfs using \ac{HCIPy}, an open-source object-oriented framework written in Python for performing end-to-end simulations of high-contrast imaging instruments \cite{hcipy}. The framework is used to generate \acp{PSF} with a very rough simulation. We use the provided methods in \ac{HCIPy} to create a multi-layer atmospheric model that changes in time. Then we use that model, the \ac{vAPP} amplitude and \ac{vAPP} phase screen to generate a series of \acp{PSF} using:

\begin{equation}
I_vAPP = \mathscr{F}\Big( Ae^{i\theta}\Big) + \mathscr{F}\Big( Ae^{i\theta}\Big) + \mathscr{F}\Big( Ae \Big) \cdot C
\end{equation}

Here we use not one apperture function ($Ae^{i\theta}$) as in \autoref{sec:PSF} but combine three. The first two simulate the opposite phase the two circular polarisations receive from the \ac{vAPP} (see \autoref{sec:vapp}). The third factor, called the leakage term ************************************************
%TODO refer to leakage term, explain leakage term in vAPP theory section.

Instead of modelling an active optics system we modified the fried parameter for the atmosphere to get similar morphological changes in time to the availible on sky images. This is sufficient as the dark hole, where the we will observes the disk, will lie in the control radius of the \ac{AO} which means the effect of an \ac{AO} system on the disk will mostly be a increase in resolution by reducing the seeing effect.

We use a telescope diamater of $8.2m$ and a wavelength, of $1\cdot 10^{-6}m$ for the generation. For a fried paramater of $4m$ we find quantativly similar psf morphologies to the on sky data. To put this into perspective: to simulate excellent seeing conditions we would use 20cm. With these setting we use \ac{HCIPy} to create an ordered set of \acp{PSF} changing through time as the simulated atmosphere evolves. See \autoref{fig:psfs_evolving} for the difference between the first \ac{PSF} in a set and later \acp{PSF} in the same set, compared between on sky and generated \acp{PSF}. Note that as time evolves the differences between first and later \acp{PSF} grow. Qualitatively on sky and generated \acp{PSF} evolve similar in time. We decide to use this method to generate all \acp{PSF}.

%https://tex.stackexchange.com/questions/169219/subcaption-third-figure-breaks-column
%https://tex.stackexchange.com/questions/139214/how-to-make-a-subfigure-span-the-two-columns-in-ieeetrans-style

\begin{figure}[h!]

  %\begin{flushright}
  
  %define (\edef) the variable \side to l or r depending if we are at an odd
  %or even page
  \checkoddpage 
  \edef\side{\ifoddpage l\else r\fi}%
  
  %\hspace{-1\linewidth}%
  \makebox[\textwidth][\side]{%
  
  \begin{minipage}[t]{1.3\textwidth}
      \begin{subfigure}[b]{0.6\textwidth}
        \includegraphics[width=\linewidth]{gfx/plots/miscellaneous/0_psf_diff_007}
        \caption{Simulated psf: 1e - 2e, 0.7 seconds difference.}
      \end{subfigure}% this comment sign needs to be here for the images to be on the same line
      \begin{subfigure}[b]{0.6\textwidth}
        \includegraphics[width=\linewidth]{gfx/plots/miscellaneous/0_on_sky_psf_diff_007}
        \caption{On sky psf: 1th - 2th, *** seconds difference.}
      \end{subfigure}      
      
      \begin{subfigure}[b]{0.6\textwidth}
        \includegraphics[width=\linewidth]{gfx/plots/miscellaneous/0_psf_diff_070}
        \caption{Simulated psf: 1th - 10th, 7 seconds difference.}
      \end{subfigure}% this comment sign needs to be here for the images to be on the same line
      \begin{subfigure}[b]{0.6\textwidth}
        \includegraphics[width=\linewidth]{gfx/plots/miscellaneous/0_on_sky_psf_diff_070}
        \caption{On sky psf: 1th - 10th, *** seconds difference.}
      \end{subfigure}
            
      \begin{subfigure}[b]{0.6\textwidth}
        \includegraphics[width=\linewidth]{gfx/plots/miscellaneous/0_psf_diff_140}
        \caption{Simulated psf: 1th - 50th, 14 seconds difference.}
      \end{subfigure}% this comment sign needs to be here for the images to be on the same line
      \begin{subfigure}[b]{0.6\textwidth}
        \includegraphics[width=\linewidth]{gfx/plots/miscellaneous/0_on_sky_psf_diff_140}
        \caption{On sky psf: 1th - 50th, *** seconds difference.}
      \end{subfigure}
  
  \end{minipage}
  }%
  %\end{flushright}

  \caption{Differences between the first psf of an observation and later psfs. Compared between simulated observations on the left and on sky data on the right. The psfs where aligned then normalised on the maximum of the leakage term, finally the absolute value of the diffrence between the first and n-th psf was taken. The second, 10th and 50th simulated psf are 0.7, 7 and 14 seconds apart in the simulation.}
  \label{fig:psfs_evolving}
\end{figure}

%----------------------------------------------------------------------------------------





\section{Generating observations}
\label{sec:gen}
To get a dataset that simulates an obeservation we create an ordered set of disk images each one rotated a bit to eachother. This accounts for the field rotation between exposures caused by observing with an alt azimuth telescope. The angle depends on the time between exposures and the field rotation rate.

\begin{align}
  \psi &= 0.2506 \cdot \frac{cos(A) cos \phi}{sin(z)} 
\end{align}

The field rotation $\psi$ in degrees per minute for a given target azimuth $A$, zenith distance $z$ and telescope latitude $\psi$ \cite[page 95]{Electronic_imaging}. If we would be observing from mauna kea (latitude 19.8) at a resonable 30 degrees from the zenith this gives a rotation rate between 0 and  14.74 degrees per minute.

Now to get the simulated observation an ordered set is created by convolving each n-th image from the disk set with the n-th image of the psf set. These are the simulated observations through time. 

Using variations on this method we create 3 different sets.

\begin{enumerate}
\item Place the star as a single pixel with value one in the center of the disk before going through the above procedure. This is our observation set.
\item Leave out the disk, place only the single pixel in the center representing the star. This can be useful for checking if data reduction algoritmes work properly.
\item Leave out the star, this way we see what the \ac{vAPP} does to the extended disk.
\end{enumerate}

For each of the sets we can generate an additional variant to look at what the best result attainable is by leaving out the psf distortions. We do this by taking the convolution of the n-th disk image with the first \ac{PSF} of the \ac{PSF} set instead of the n-th.
