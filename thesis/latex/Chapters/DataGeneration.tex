% Chapter X

\chapter{Data generation} % Chapter title
\label{ch:data_gen} % For referencing the chapter elsewhere, use \autoref{ch:name} 

There are no observations of disks with the \ac{vAPP}. Here we describe how we simulate observations using a disk model and \acp{PSF}. This presents a number of advantages though the main reason is the lack of disk observations using the \ac{vAPP}. Generating the data from a model allows us to vary parameters as we wish. We can try a reduction method on a simple slightly inclined disk or a near face-on disk with many rings. By reducing images with varying disk morphology we can map how a reduction method performs for each disk morphology. Most importantly we can clearly separate artifacts created during data reduction from disk features when we know what the disk looks like.

%------------------------------------------------



\section{Disk model}
We use a 2-dimensional thin disk model based on the work by \cite{Pieter_Okko}. The model has 4 basic parameters: 

\begin{description}
\item[Inclination] The angle the disk is tilted towards the observer. A 0 degrees inclination gives a face-on disk and 90 degrees a horizontal line being an edge on disk. As illustrated in \autoref{fig:inclination}
\item[Position angle] After inclination the disk can be rotated around the line of sight from the observer, rotation to the left is positive. See \autoref{fig:pos_angle}
\item[Inner and Outer radii] Disks consist of one or more rings that start and stop at some radius from the star. The inner and outer radius are relative to the size of the image generated and expressed in percent. An Inner radius of 20\% gives a hole in the disk with a diameter of 20\% of the image width. %
\end{description}

\begin{figure}[h!]
  \centering
  \begin{subfigure}[b]{0.4\textwidth}
    \includegraphics[width=\textwidth]{gfx/inclination}
    \caption{Inclination angle from: \cite{Pieter_Okko}}
    \label{fig:pos_angle}

  \end{subfigure}
  \begin{subfigure}[b]{0.4\textwidth}
    \includegraphics[width=\textwidth]{gfx/pos_angle}
    \caption{Position angle, from: \cite{Pieter_Okko}}
    \label{fig:inclination}
  \end{subfigure}
  \label{fig:1}
\end{figure}

The disk is modeled as optically thick and does not emit light on its own. At a certain radii between the inner and outer radii ($r$) the disks brightness profile is given by:

\begin{equation}
B(r) = 1 \cdot {\left(\frac{R_{\text{star}}}{r}\right)}^2
\end{equation}

Here $B(r)$ is the value of a disk pixel at distance $r$ in AU and $R_\text{sun}$ is the suns radius. The disk pixel values are increased until the required contrast between star and disk is reached. For all disks discussed here except otherwise noted this factor is $50$.

We evaluate the model onto an image of 200 by 200 pixels. This speeds up calculations and is similar to the resolution of observed \ac{vAPP} data-sets. The model behaves well for most disks however on this resolution features that are small in the observers plane are pixelated. We can make disks larger then the full image width if they are inclined and rotated as we see in \autoref{fig:disk} below..

\begin{figure}[h!]
  \begin{subfigure}[t]{0.5\textwidth}
    \includegraphics[width=1.2\linewidth]{gfx/plots/miscellaneous/8_disk_45_60}
    \caption{Model output for an inclination of $60\deg$, an inner radius of $20\%$ and an outer radius of $50\%$.}
  \end{subfigure}
  \begin{subfigure}[t]{0.5\textwidth}
    \includegraphics[width=1.2\linewidth]{gfx/plots/miscellaneous/8_disk_45_10_22_23}
    \caption{Model output for an inclination of $80\deg$, an outer ring from $100\%$ to $130\%$ and an inner ring from $30\%$ to $80\%$.}
  \end{subfigure}
  \caption{Two disks created by the model with different rings and inclination both with a position angle of 45 degrees.}
  \label{fig:disk}
\end{figure}

%------------------------------------------------




\section{Atmospheric Distortions}

To simulate an observation we do not use a single given \ac{PSF}. Since during an observation the combined \ac{PSF}  of the atmosphere and instrument changes. However a complete simulation of these effects is outside the scope of the thesis. Instead we try to approach a similar morphology to an on-sky-\ac{PSF}, with similar changes in time. We try two different methods, adding distortions on top of an existing \ac{vAPP} \ac{PSF} and adding an atmosphere simulation to the PSF generation. Finally we compare the result of each method to on-sky observations.

\subsection{Fourier method}
First we tried modifying a single given \ac{vAPP} \ac{PSF} to get a set of disturbed \acp{PSF}. To achieve this a pattern is added to the Fourier transform of the \ac{PSF} before transforming it back from Fourier space, see \autoref{eq:Ifinal} below. The pattern is then shifted for every timestamp.

\begin{equation}
I_{\text{final}} = \mathscr{F}_{2d}^{-1} \Big( \text{intensity} \cdot \mathscr{F}_{2d}\left(\text{\ac{PSF}}\right) \cdot \text{pattern} + \left(1-\text{intensity}\right) \cdot \mathscr{F}_{2d}\left(\text{\ac{PSF}}\right) \Big) \label{eq:Ifinal}
\end{equation}
Here "intensity" is a number between $0$ and $1$, $\mathscr{F}_{2d}^{-1}$ the inverted 2d Fourier transform and $\mathscr{F}_{2d}$ the normal 2d Fourier transform.\\

The best results where achieved using a grid of blurred circles as pattern, see \autoref{fig:patterns}. Note the distortions to the \ac{PSF} are around the center of its peaks but not randomly spread. As clearly this clearly does not suffice to simulate realistic distortions.

\begin{figure}[h!]
      \begin{subfigure}[t]{0.5\textwidth}
        \includegraphics[width=1.2\linewidth]{gfx/0_old_psf_diff1}
        \caption{Difference between two \acp{PSF}  created using this method, between \acp{PSF} the pattern has shifted 10 pixels.}
      \end{subfigure}% this comment sign needs to be here for the images to be on the same line
      \begin{subfigure}[t]{0.5\textwidth}
        \includegraphics[width=1.2\linewidth]{gfx/0_old_psf_diff2}
        \caption{Difference between two \acp{PSF}  created using this method, between \acp{PSF} the pattern has shifted 5 pixels.}
      \end{subfigure}
      
      \begin{subfigure}[]{0.5\textwidth}
        \includegraphics[width=1.2\linewidth]{gfx/0_pattern}
        \caption{Pattern that is shifted and applied to the Fourier space of a \ac{PSF}.}
      \end{subfigure}
      
  \caption{The Pattern used and the differences between \acp{PSF} distorted with it using the method described above.}
  \label{fig:patterns}
\end{figure}
%TODO [optional] add psfs to how how shift happens

\subsection{Using HCIPy}
\label{sec:hcipy}
Then we tried a simulation of the  \acp{PSF} using \ac{HCIPy}, an open-source object-oriented framework written in Python for performing end-to-end simulations of high-contrast imaging instruments \cite{hcipy}. The framework is used to generate \acp{PSF} with a very rough simulation. We use the provided methods in \ac{HCIPy} to create a multi-layer atmospheric model that changes in time. Then we use that model, the \ac{vAPP} amplitude and \ac{vAPP} phase screen to generate a series of \acp{PSF} using:

\begin{equation}
I_\text{\ac{vAPP}} = {\mathopen| \mathscr{F}_{2d}\Big( Ae^{i\theta}\Big) \mathclose|}^2 + {\mathopen| \mathscr{F}_{2d}\Big( Ae^{-i\theta}\Big) \mathclose|}^2 + {\mathopen| \mathscr{F}_{2d}\Big( A \Big) \cdot C \mathclose|}^2
\end{equation}
With $I_\text{\ac{vAPP}}$ the \ac{vAPP} \ac{PSF}, $\mathscr{F}_{2d}$ the 2 dimensional Fourier transformation. $A$ and $e^{i\theta}$ represent an amplitude and phase change imparted on the light by the \ac{vAPP}.\\

Here we use not one aperture function as in \autoref{sec:PSF} but combine three. The first two ($Ae^{\pm i\theta}$) simulate the opposing phases the circular polarization receive from the \ac{vAPP} (see \autoref{fig:into_vapp_annotated}). The third factor, is called the leakage term.
%TODO [optional] refer to leakage term, explain leakage term in vAPP theory section.

Instead of modeling an adaptive optics system we modified the Fried parameter for the atmosphere to get similar morphological changes in time to the available on-sky images. This is sufficient as the dark hole, where we will observes the disk, will lie in the control radius of the \ac{AO} which means the effect of an \ac{AO} system on the disk will mostly be a increase in resolution by reducing the seeing effect.
%TODO check the above sentence with Steven

%TODO check with steven what dataset I got from him, ask how to cite datasets then cite the dataset
We use a telescope diameter of $8.2m$ and a wavelength, of $1\cdot 10^{-6}m$ for the generation. For a Fried parameter of $4m$ and exposure time of $0.7$ seconds we find quantitatively similar \ac{PSF} mythologies to the on-sky data\footnote{Unpublished data taken with the SCExAO instrument on the Subaru telescope during an engineering night in the summer of 2017.}. To put this into perspective: to simulate excellent seeing conditions we would use a Fried parameter of 20cm. With these setting we use \ac{HCIPy} to create an ordered set of \acp{PSF} changing through time as the simulated atmosphere evolves. See \autoref{fig:psfs_evolving} for the difference between the first \ac{PSF} in a set and later \acp{PSF} in the same set, compared between on-sky and simulated \acp{PSF}. Note that as time evolves the differences between first and later \acp{PSF} grow. The \textquote{brightness} of the differences can not be compared as the on sky data saturates the detector for most of the coronagraphic \ac{PSF}. Qualitatively on-sky and generated \acp{PSF} evolve similar in time. We decide to use this method to generate all \acp{PSF}.

%TODO [critical] add that we use unrealistic exposure time throughout the thesis to simulate correct morphology, also add staturation explanation for wierd values on colorbar

%https://tex.stackexchange.com/questions/169219/subcaption-third-figure-breaks-column
%https://tex.stackexchange.com/questions/139214/how-to-make-a-subfigure-span-the-two-columns-in-ieeetrans-style

\begin{figure}[h!]
  %define (\edef) the variable \side to l or r depending if we are at an odd
  %or even page
  \checkoddpage 
  \edef\side{\ifoddpage l\else r\fi}%
  
  \makebox[\textwidth][\side]{%
  
  \begin{minipage}[t]{1.3\textwidth}
      \begin{subfigure}[b]{0.6\textwidth}
        \includegraphics[width=\linewidth]{gfx/plots/miscellaneous/0_psf_diff_007}
        \caption{Simulated \ac{PSF}: 1e - 2e, 0.7 seconds difference.}
      \end{subfigure}% this comment sign needs to be here for the images to be on the same line
      \begin{subfigure}[b]{0.6\textwidth}
        \includegraphics[width=\linewidth]{gfx/plots/miscellaneous/0_on_sky_psf_diff_007}
        \caption{On sky \ac{PSF}: 1th - 2th, 0.0014 seconds difference.}
      \end{subfigure}      
      
      \begin{subfigure}[b]{0.6\textwidth}
        \includegraphics[width=\linewidth]{gfx/plots/miscellaneous/0_psf_diff_070}
        \caption{Simulated \ac{PSF}: 1th - 10th, 7 seconds difference.}
      \end{subfigure}% this comment sign needs to be here for the images to be on the same line
      \begin{subfigure}[b]{0.6\textwidth}
        \includegraphics[width=\linewidth]{gfx/plots/miscellaneous/0_on_sky_psf_diff_070}
        \caption{On sky \ac{PSF}: 1th - 10th, 0.014 seconds difference.}
      \end{subfigure}
            
      \begin{subfigure}[b]{0.6\textwidth}
        \includegraphics[width=\linewidth]{gfx/plots/miscellaneous/0_psf_diff_140}
        \caption{Simulated \ac{PSF}: 1th - 50th, 14 seconds difference.}
      \end{subfigure}% this comment sign needs to be here for the images to be on the same line
      \begin{subfigure}[b]{0.6\textwidth}
        \includegraphics[width=\linewidth]{gfx/plots/miscellaneous/0_on_sky_psf_diff_140}
        \caption{On sky \ac{PSF}: 1th - 50th, 0.07 seconds difference.}
      \end{subfigure}
  
  \end{minipage}
  }%

  \caption{Differences between the first  \ac{PSF}  of an observation and later \acp{PSF}. Compared between simulated observations on the left and on-sky data on the right. The  \acp{PSF}  where aligned then normalized on the maximum of the leakage term, finally the absolute value of the difference between the first and n-th  \ac{PSF}  was taken. The second, 10th and 50th simulated  \ac{PSF} are 0.7, 7 and 14 seconds apart in simulation time. These give similar morphology to the on sky differences that are 0.0014, 0.014 and 0.07 seconds apart.}
  \label{fig:psfs_evolving}
\end{figure}
%TODO [optional] add unaltered psfs in appendix

%----------------------------------------------------------------------------------------





\section{Generating observations}
\label{sec:gen}
To get a data-set that simulates an observation we create an ordered set of disk images each one rotated a bit with respect to the previous image. This accounts for the field rotation between images caused by observing with an alt-azimuth telescope in pupil-stabilized mode. The angle depends on the time between images and the field rotation rate.

\begin{align}
  \psi &= 0.2506 \cdot \frac{\cos(A) \cos \phi}{\sin(z)} 
\end{align}
%TODO discuss with Jos, book page 95 agrees with whats here

The field rotation $\psi$ in degrees per minute for a given target azimuth $A$, zenith distance $z$ and telescope latitude $\psi$ \cite[page 95]{Electronic_imaging}. If we would be observing from Mauna Kea (latitude 19.8) at 30 degrees from the zenith this gives a rotation rate between 0 and 14.74 degrees per minute.%TODO todo recalculate result after discussing with steven

Now to get the simulated observation an ordered set is created by convolving each n-th image from the disk set with the n-th image of the  \ac{PSF}  set. These are the simulated observations through time. 

Using variations on this method we create 3 different sets.

\begin{enumerate}
\item Place the star as a single pixel with value one in the center of the disk before going through the above procedure. This is our observation set.
\item Leave out the disk, place only the single pixel in the center representing the star. This can be useful for checking if data reduction algorithms work properly.
\item Leave out the star, this way we see what the \ac{vAPP} does to the extended disk.
\end{enumerate}

For each of the sets we can generate an additional variant to look at what the best result attainable is by leaving out the  \ac{PSF}  distortions. We do this by taking the convolution of the n-th disk image with the first \ac{PSF} of the \ac{PSF} set instead of the n-th.

We do not try to simulate noise sources as detector or photon noise. These will not affect the disks morphology and the techniques for handling them are the same between disks and other objects.
