% principle component analysis + adi on disks: https://arxiv.org/pdf/1905.01860.pdf
\chapter{Discussion and Conclusion}

Here we discuss the results and conclude wheter looking at disks reduced with \ac{ADI} from \ac{vAPP} data shows promise. We first take a look at what the \ac{vAPP} and \ac{ADI} do to the disk, see the first two figures in \autoref{chap:results}. Then we compare different disk models and the result after \ac{ADI}. We see if we can recognise the original morphology and or we can clearly seperate the different models based on the results. 

\section{ADI disk results in depth}
Two disks \textit{A} and \textit{B} (see \autoref{tab:disks}) are presented in depth, they are identical except one has two rings and the other only one. Looking at the results for the first disk we can just make out the disk in the simulated image in \autoref{fig:disk4_simulated}. If we look at \autoref{fig:disk4_nostar} showing only the light reflected by the disk we see that the \ac{vAPP} has smeared out the light reflected by the disk, the morphology has not been distorted. When \ac{ADI} is applied we see how the disk morphology is changed. We see (\autoref{fig:disk4_nostar_adi}) that the morphology is radically changed quite, a number of features appear. When \ac{ADI} is applied to the simulated image (see \autoref{fig:disk4_adi}) we see residuals from the \ac{PSF} overlaying the morphologically changed disk in the dark hole. Note that the changed disk is slightly smeared out and faded compared to \autoref{fig:disk4_nostar_adi}. This is caused by the atmosphere changing the \acp{PSF} and by the \ac{ADI} process.
    Moving on to the second disk (\textit{B}), we find the same as for the first disk. However this disk has a small inner ring, if we look at \autoref{fig:disk0_simulated} we see that only a small portion of the inner disk lies in the dark zone. We see in \autoref{fig:disk0_nostar_adi} that the light of the inner ring is concentrated into blobs at opposing sides of the rings semi major axis. When we look at the result of \ac{ADI} on the simulated image of disk \textit{B} (see \autoref{fig:disk0_adi}) we can recognise the blob. 
    We conclude that \ac{ADI} can help us differentiate different disk. 

\section{Characterising paramaters in ADI results}
The distortions to the disks morphology created by \ac{ADI} might allow characterisation of the disks paramaters.



%The single ring of disk \textit{A} can be recognised easily in \autoref{fig:disk4_simulated} while the result after \ac{ADI} in \autoref{fig:disk4_adi} ****TODO improve plot \ make it logplot, add histogram of pixels along semi major axis*****. 

%We see that the morphology of the disks is heavily impacted by the \ac{ADI} process when looking at the images of the \textit{starless} model variants. While for disk A the morphology can be recognised in the dark zone of the \ac{vAPP}. 

%When we look at disk \textit{B} which features 2 rings we notice the seperation between the rings can be recognised from the blob we see on its semi-major axis.  

%However the distortions to the disks morphology created by \ac{ADI} might allow easy characterisation of the disks paramaters. And \ac{ADI} results might still be improved with different techniques, see \autoref{chap:future}.

%conclusions from the direct comparisions
%-lowering contrast does not impact the view of the disk in the dark hole a lot
%-lowering contrast does impact the ADI result significantly

\section{Characterising features in ADI results}

Besides the already dicussed brightness comparision we have looked at the effect of adding a ring, scaling the disk size, the orientation of the disk and lowering the inclination. We will be comparing the \ac{ADI} results, these are the simuleted images for a model and its \textit{starless} variant after processing with our \ac{ADI} implementation.

Adding an extra ring to \textit{disk E} turns it into \textit{disk B}. Comparing the results of both we see no changes to the outer ring, it is still present in both models and in the result of both models. **** TODO we expect this because: refer back to earlier part of thesis about liniarity****. The innner ring of \textit{disk E} in \autoref{fig:comp_numb_rings} can be identified in the \ac{ADI} results. Hoever not as a circular ring, though that is present, but by two mirrored blobs on its semi major axis. If we look at \textit{disk F} (see: \autoref{fig:disk3_1}, \autoref{fig:disk3_2}) which has a similarly sized inner ring we see the same blobs at the same place. The blobs are not a distortion from the double rings but from the tight inner ring. 

conclusions from the in depth look
-closer to the center of disk => more side blobs on semi major axis

-a ring is detectable in the adi result as a gap peak to the left of the coronografic \ac{PSF}, easiest to see between bulb and outer disk on disk major axis

-placeholder for large vs small disk (need to check if the disk does not rotate out of the frame

-a disk that rotates out of the drak hole during \ac{ADI} loses contrast on one side (check which with the appendix)

-lower inclination gives a rounder disk

from above subconclutions to conclusion:



