% Chapter X

\chapter{Introduction} % Chapter title

\label{ch:intro} % For referencing the chapter elsewhere, use \autoref{ch:name} 

%----------------------------------------------------------------------------------------

A major and successful topic in astronomy has been the discovery and characterization of exoplanets. This interest has sparked great progress in the field of high contrast imaging resulting in new imaging techniques and instruments. Though as of writing $3,949$ planets have been confirmed little is known about their formation. Certain is that a so called protoplanetary disk is an important stage in the formation of a planetary system. The disks can be observed best within the visible spectrum and infrared though ALMA has also had success in observing disks in the radio spectrum. 

It is however a great challenge to acquire observations of such a disk due to the high contrast with its star and the required angular resolution. To observe with sufficient resolution adaptive optics are used. This negates most of the atmospheric seeing.
%TODO ****explain why light from a star (dot) can hide planetary details in a normal telescope (is it only sys+atmosph psf for apparture?)****

Reducing contrast is done with chronograph, an instrument that blocks out the direct light from a star. The classic Lyot chronograph (\autoref{fig:lyot_sketch}) blocks direct star light using two foci. In the first focus an opaque mask blocks placed where the star is diffuses and absorbs the direct star light. Then between the foci a ring shaped mask blocks most of the now diffused star light. Then at the second focus the image is made as usual.

\begin{figure}
    \includegraphics[width= 1\textwidth]{gfx/lyot_sketch}
    \caption{optical layout of a Lyot chronograph, by Matthew Kenworthy, from https://home.strw.leidenuniv.nl/~kenworthy/app}
    \label{fig:lyot_sketch}
\end {figure}

%TODO what paper to cite for vapp?
However a classical chronograph is not sufficient for observing disks. The \ac{vAPP} \cite{papervApp} is a different type of chronograph placed at the pupil of a telescope. The \ac{vAPP} blocks starlight by modifying the phase of incoming light. This phase change flips the light in a D-shaped region around the star to the other side. Any faint object next to the star in the now dark region becomes detectable. Compared to the classic Lyot chronograph the \ac{vAPP} reduces the starlight to a greater degree, see \autoref{fig:vapp_vs_lyot}.  

\begin{figure}[h!]
      \begin{subfigure}[t]{0.5 \textwidth}
        \includegraphics[width=1.2\linewidth]{gfx/plots/lyot/0_lyot_psf}
        \label{fig:lyot}
        \caption{PSF of a classical Lyot chronograph}
      \end{subfigure}% this comment sign needs to be here for the images to be on the same line
      \begin{subfigure}[t]{0.5\textwidth}
        \includegraphics[width=1.2\linewidth]{gfx/plots/lyot/0_normal_psf}
        \caption{PSF of a telescope without chronograph}
        \label{fig:classic_psf}
      \end{subfigure}
      
      \begin{subfigure}[]{0.5\textwidth}
        \includegraphics[width=1.2\linewidth]{gfx/plots/miscellaneous/0_clean_vapp}
        \caption{PSF of the vAPP}
      \end{subfigure}
      
  \caption{The \ac{PSF} is the image an instrument produces when looking at a point source such as a star. Here we sees the \ac{PSF} of a instrument without a chronograph, with the classical Lyot chronograph and a one using the vAPP }
  \label{fig:vapp_vs_lyot}
\end{figure}

The \ac{vAPP} has been developed to detect rocky planets in the habitable zone of stars. However it could also allow us to resolve disk features and study disks in greater detail directly. However as the \ac{vAPP} changes the entire image it is chalking to differentiate disk features from \ac{vAPP} artifacts. 

%improve
Here we study what the effect is of both the \ac{vAPP} and \ac{ADI} on the apparent morphology of disks.
%improve


%***TODO***
overview thesis


%include image from: https://home.strw.leidenuniv.nl/~kenworthy/app
https://en.wikipedia.org/wiki/Coronagraph
