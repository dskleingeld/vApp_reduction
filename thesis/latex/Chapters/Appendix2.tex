\begin{figure}[!hbt]
  \newcommand{\plotId}{C}
  \newcommand{\plotside}{right}
  %define (\edef) the variable \side to l or r depending if we are at an odd
  %or even page
  \checkoddpage 
  \edef\side{\ifoddpage l\else r\fi}%
  
  \makebox[\textwidth][\side]{%
  
  \begin{minipage}[t]{1.2\textwidth}
      \begin{subfigure}[t]{0.6\textwidth}
        \includegraphics[width=\linewidth]{gfx/plots/results/\plotId/\plotside_coro_psf}
        \caption[]{The coronografic \ac{PSF} used to create c d e and f, the brightness is relative to the center star}
      \end{subfigure}% this comment sign needs to be here for the images to be on the same line      
      \begin{subfigure}[t]{0.6\textwidth}
        \includegraphics[width=\linewidth]{gfx/plots/results/\plotId/model}
        \caption[]{The disk model, the brightness is relative to the center star.}
      \end{subfigure}
      
      \begin{subfigure}[t]{0.6\textwidth}
        \includegraphics[width=\linewidth]{gfx/plots/results/\plotId/\plotside_noReductin}
        \caption[]{A simulated image of the disk model. We can see directly in the image where the disk lies.}
      \end{subfigure}% this comment sign needs to be here for the images to be on the same line      
      \begin{subfigure}[t]{0.6\textwidth}
        \includegraphics[width=\linewidth]{gfx/plots/results/\plotId/\plotside_ADI}
        \caption[]{\ac{ADI} applied simulated image (left). It is hard to discern disk features from \ac{PSF} residuals.}
      \end{subfigure}
      
      \begin{subfigure}[t]{0.6\textwidth}
        \centering
        \includegraphics[width=\linewidth]{gfx/plots/results/\plotId/\plotside_no_star_noReductin}
        \caption[]{A simulated image of the \textit{starless} variant of the disk model. We see where the light of the disk gets spread by the PSF. The disks two rings can clearly be distinguished}
      \end{subfigure}% this comment sign needs to be here for the images to be on the same line
      \begin{subfigure}[t]{0.6\textwidth}
        \centering
        \includegraphics[width=\linewidth]{gfx/plots/results/\plotId/\plotside_no_star_ADI}
        \caption[]{\ac{ADI} applied simulated image of the \textit{starless} variant (left). The outer disk can still be recognized though various gaps have appeared. The inner disk has become really thin and has two large blobs at opposite side on its semi major axis.}
      \end{subfigure}
  \end{minipage}
  }%

  \caption[]{On the top row: the \ac{PSF} and the disk model used to generate the other images. Then generated images of the disk model en its \textit{starless} variant (left) and results of \ac{ADI} applied to them (right).}
  \label{fig:\plotside:apd\plotId}
\end{figure}

\begin{figure}[!hbt]
  \newcommand{\plotId}{D}
  \newcommand{\plotside}{right}
  %define (\edef) the variable \side to l or r depending if we are at an odd
  %or even page
  \checkoddpage 
  \edef\side{\ifoddpage l\else r\fi}%
  
  \makebox[\textwidth][\side]{%
  
  \begin{minipage}[t]{1.2\textwidth}
      \begin{subfigure}[t]{0.6\textwidth}
        \includegraphics[width=\linewidth]{gfx/plots/results/\plotId/\plotside_coro_psf}
        \caption[]{The coronografic \ac{PSF} used to create c d e and f, the brightness is relative to the center star}
      \end{subfigure}% this comment sign needs to be here for the images to be on the same line      
      \begin{subfigure}[t]{0.6\textwidth}
        \includegraphics[width=\linewidth]{gfx/plots/results/\plotId/model}
        \caption[]{The disk model, the brightness is relative to the center star.}
      \end{subfigure}
      
      \begin{subfigure}[t]{0.6\textwidth}
        \includegraphics[width=\linewidth]{gfx/plots/results/\plotId/\plotside_noReductin}
        \caption[]{A simulated image of the disk model. We can see directly in the image where the disk lies.}
      \end{subfigure}% this comment sign needs to be here for the images to be on the same line      
      \begin{subfigure}[t]{0.6\textwidth}
        \includegraphics[width=\linewidth]{gfx/plots/results/\plotId/\plotside_ADI}
        \caption[]{\ac{ADI} applied simulated image (left). It is hard to discern disk features from \ac{PSF} residuals.}
      \end{subfigure}
      
      \begin{subfigure}[t]{0.6\textwidth}
        \centering
        \includegraphics[width=\linewidth]{gfx/plots/results/\plotId/\plotside_no_star_noReductin}
        \caption[]{A simulated image of the \textit{starless} variant of the disk model. We see where the light of the disk gets spread by the PSF. The disks two rings can clearly be distinguished}
      \end{subfigure}% this comment sign needs to be here for the images to be on the same line
      \begin{subfigure}[t]{0.6\textwidth}
        \centering
        \includegraphics[width=\linewidth]{gfx/plots/results/\plotId/\plotside_no_star_ADI}
        \caption[]{\ac{ADI} applied simulated image of the \textit{starless} variant (left). The outer disk can still be recognized though various gaps have appeared. The inner disk has become really thin and has two large blobs at opposite side on its semi major axis.}
      \end{subfigure}
  \end{minipage}
  }%

  \caption[]{On the top row: the \ac{PSF} and the disk model used to generate the other images. Then generated images of the disk model en its \textit{starless} variant (left) and results of \ac{ADI} applied to them (right).}
  \label{fig:\plotside:apd\plotId}
\end{figure}

\begin{figure}[!hbt]
  \newcommand{\plotId}{E}
  \newcommand{\plotside}{right}
  %define (\edef) the variable \side to l or r depending if we are at an odd
  %or even page
  \checkoddpage 
  \edef\side{\ifoddpage l\else r\fi}%
  
  \makebox[\textwidth][\side]{%
  
  \begin{minipage}[t]{1.2\textwidth}
      \begin{subfigure}[t]{0.6\textwidth}
        \includegraphics[width=\linewidth]{gfx/plots/results/\plotId/\plotside_coro_psf}
        \caption[]{The coronografic \ac{PSF} used to create c d e and f, the brightness is relative to the center star}
      \end{subfigure}% this comment sign needs to be here for the images to be on the same line      
      \begin{subfigure}[t]{0.6\textwidth}
        \includegraphics[width=\linewidth]{gfx/plots/results/\plotId/model}
        \caption[]{The disk model, the brightness is relative to the center star.}
      \end{subfigure}
      
      \begin{subfigure}[t]{0.6\textwidth}
        \includegraphics[width=\linewidth]{gfx/plots/results/\plotId/\plotside_noReductin}
        \caption[]{A simulated image of the disk model. We can see directly in the image where the disk lies.}
      \end{subfigure}% this comment sign needs to be here for the images to be on the same line      
      \begin{subfigure}[t]{0.6\textwidth}
        \includegraphics[width=\linewidth]{gfx/plots/results/\plotId/\plotside_ADI}
        \caption[]{\ac{ADI} applied simulated image (left). It is hard to discern disk features from \ac{PSF} residuals.}
      \end{subfigure}
      
      \begin{subfigure}[t]{0.6\textwidth}
        \centering
        \includegraphics[width=\linewidth]{gfx/plots/results/\plotId/\plotside_no_star_noReductin}
        \caption[]{A simulated image of the \textit{starless} variant of the disk model. We see where the light of the disk gets spread by the PSF. The disks two rings can clearly be distinguished}
      \end{subfigure}% this comment sign needs to be here for the images to be on the same line
      \begin{subfigure}[t]{0.6\textwidth}
        \centering
        \includegraphics[width=\linewidth]{gfx/plots/results/\plotId/\plotside_no_star_ADI}
        \caption[]{\ac{ADI} applied simulated image of the \textit{starless} variant (left). The outer disk can still be recognized though various gaps have appeared. The inner disk has become really thin and has two large blobs at opposite side on its semi major axis.}
      \end{subfigure}
  \end{minipage}
  }%

  \caption[]{On the top row: the \ac{PSF} and the disk model used to generate the other images. Then generated images of the disk model en its \textit{starless} variant (left) and results of \ac{ADI} applied to them (right).}
  \label{fig:\plotside:apd\plotId}
\end{figure}

\begin{figure}[!hbt]
  \newcommand{\plotId}{F}
  \newcommand{\plotside}{right}
  %define (\edef) the variable \side to l or r depending if we are at an odd
  %or even page
  \checkoddpage 
  \edef\side{\ifoddpage l\else r\fi}%
  
  \makebox[\textwidth][\side]{%
  
  \begin{minipage}[t]{1.2\textwidth}
      \begin{subfigure}[t]{0.6\textwidth}
        \includegraphics[width=\linewidth]{gfx/plots/results/\plotId/\plotside_coro_psf}
        \caption[]{The coronografic \ac{PSF} used to create c d e and f, the brightness is relative to the center star}
      \end{subfigure}% this comment sign needs to be here for the images to be on the same line      
      \begin{subfigure}[t]{0.6\textwidth}
        \includegraphics[width=\linewidth]{gfx/plots/results/\plotId/model}
        \caption[]{The disk model, the brightness is relative to the center star.}
      \end{subfigure}
      
      \begin{subfigure}[t]{0.6\textwidth}
        \includegraphics[width=\linewidth]{gfx/plots/results/\plotId/\plotside_noReductin}
        \caption[]{A simulated image of the disk model. We can see directly in the image where the disk lies.}
      \end{subfigure}% this comment sign needs to be here for the images to be on the same line      
      \begin{subfigure}[t]{0.6\textwidth}
        \includegraphics[width=\linewidth]{gfx/plots/results/\plotId/\plotside_ADI}
        \caption[]{\ac{ADI} applied simulated image (left). It is hard to discern disk features from \ac{PSF} residuals.}
      \end{subfigure}
      
      \begin{subfigure}[t]{0.6\textwidth}
        \centering
        \includegraphics[width=\linewidth]{gfx/plots/results/\plotId/\plotside_no_star_noReductin}
        \caption[]{A simulated image of the \textit{starless} variant of the disk model. We see where the light of the disk gets spread by the PSF. The disks two rings can clearly be distinguished}
      \end{subfigure}% this comment sign needs to be here for the images to be on the same line
      \begin{subfigure}[t]{0.6\textwidth}
        \centering
        \includegraphics[width=\linewidth]{gfx/plots/results/\plotId/\plotside_no_star_ADI}
        \caption[]{\ac{ADI} applied simulated image of the \textit{starless} variant (left). The outer disk can still be recognized though various gaps have appeared. The inner disk has become really thin and has two large blobs at opposite side on its semi major axis.}
      \end{subfigure}
  \end{minipage}
  }%

  \caption[]{On the top row: the \ac{PSF} and the disk model used to generate the other images. Then generated images of the disk model en its \textit{starless} variant (left) and results of \ac{ADI} applied to them (right).}
  \label{fig:\plotside:apd\plotId}
\end{figure}

\begin{figure}[!hbt]
  \newcommand{\plotId}{G}
  \newcommand{\plotside}{right}
  %define (\edef) the variable \side to l or r depending if we are at an odd
  %or even page
  \checkoddpage 
  \edef\side{\ifoddpage l\else r\fi}%
  
  \makebox[\textwidth][\side]{%
  
  \begin{minipage}[t]{1.2\textwidth}
      \begin{subfigure}[t]{0.6\textwidth}
        \includegraphics[width=\linewidth]{gfx/plots/results/\plotId/\plotside_coro_psf}
        \caption[]{The coronografic \ac{PSF} used to create c d e and f, the brightness is relative to the center star}
      \end{subfigure}% this comment sign needs to be here for the images to be on the same line      
      \begin{subfigure}[t]{0.6\textwidth}
        \includegraphics[width=\linewidth]{gfx/plots/results/\plotId/model}
        \caption[]{The disk model, the brightness is relative to the center star.}
      \end{subfigure}
      
      \begin{subfigure}[t]{0.6\textwidth}
        \includegraphics[width=\linewidth]{gfx/plots/results/\plotId/\plotside_noReductin}
        \caption[]{A simulated image of the disk model. We can see directly in the image where the disk lies.}
      \end{subfigure}% this comment sign needs to be here for the images to be on the same line      
      \begin{subfigure}[t]{0.6\textwidth}
        \includegraphics[width=\linewidth]{gfx/plots/results/\plotId/\plotside_ADI}
        \caption[]{\ac{ADI} applied simulated image (left). It is hard to discern disk features from \ac{PSF} residuals.}
      \end{subfigure}
      
      \begin{subfigure}[t]{0.6\textwidth}
        \centering
        \includegraphics[width=\linewidth]{gfx/plots/results/\plotId/\plotside_no_star_noReductin}
        \caption[]{A simulated image of the \textit{starless} variant of the disk model. We see where the light of the disk gets spread by the PSF. The disks two rings can clearly be distinguished}
      \end{subfigure}% this comment sign needs to be here for the images to be on the same line
      \begin{subfigure}[t]{0.6\textwidth}
        \centering
        \includegraphics[width=\linewidth]{gfx/plots/results/\plotId/\plotside_no_star_ADI}
        \caption[]{\ac{ADI} applied simulated image of the \textit{starless} variant (left). The outer disk can still be recognized though various gaps have appeared. The inner disk has become really thin and has two large blobs at opposite side on its semi major axis.}
      \end{subfigure}
  \end{minipage}
  }%

  \caption[]{On the top row: the \ac{PSF} and the disk model used to generate the other images. Then generated images of the disk model en its \textit{starless} variant (left) and results of \ac{ADI} applied to them (right).}
  \label{fig:\plotside:apd\plotId}
\end{figure}

\begin{figure}[!hbt]
  \newcommand{\plotId}{H}
  \newcommand{\plotside}{right}
  %define (\edef) the variable \side to l or r depending if we are at an odd
  %or even page
  \checkoddpage 
  \edef\side{\ifoddpage l\else r\fi}%
  
  \makebox[\textwidth][\side]{%
  
  \begin{minipage}[t]{1.2\textwidth}
      \begin{subfigure}[t]{0.6\textwidth}
        \includegraphics[width=\linewidth]{gfx/plots/results/\plotId/\plotside_coro_psf}
        \caption[]{The coronografic \ac{PSF} used to create c d e and f, the brightness is relative to the center star}
      \end{subfigure}% this comment sign needs to be here for the images to be on the same line      
      \begin{subfigure}[t]{0.6\textwidth}
        \includegraphics[width=\linewidth]{gfx/plots/results/\plotId/model}
        \caption[]{The disk model, the brightness is relative to the center star.}
      \end{subfigure}
      
      \begin{subfigure}[t]{0.6\textwidth}
        \includegraphics[width=\linewidth]{gfx/plots/results/\plotId/\plotside_noReductin}
        \caption[]{A simulated image of the disk model. We can see directly in the image where the disk lies.}
      \end{subfigure}% this comment sign needs to be here for the images to be on the same line      
      \begin{subfigure}[t]{0.6\textwidth}
        \includegraphics[width=\linewidth]{gfx/plots/results/\plotId/\plotside_ADI}
        \caption[]{\ac{ADI} applied simulated image (left). It is hard to discern disk features from \ac{PSF} residuals.}
      \end{subfigure}
      
      \begin{subfigure}[t]{0.6\textwidth}
        \centering
        \includegraphics[width=\linewidth]{gfx/plots/results/\plotId/\plotside_no_star_noReductin}
        \caption[]{A simulated image of the \textit{starless} variant of the disk model. We see where the light of the disk gets spread by the PSF. The disks two rings can clearly be distinguished}
      \end{subfigure}% this comment sign needs to be here for the images to be on the same line
      \begin{subfigure}[t]{0.6\textwidth}
        \centering
        \includegraphics[width=\linewidth]{gfx/plots/results/\plotId/\plotside_no_star_ADI}
        \caption[]{\ac{ADI} applied simulated image of the \textit{starless} variant (left). The outer disk can still be recognized though various gaps have appeared. The inner disk has become really thin and has two large blobs at opposite side on its semi major axis.}
      \end{subfigure}
  \end{minipage}
  }%

  \caption[]{On the top row: the \ac{PSF} and the disk model used to generate the other images. Then generated images of the disk model en its \textit{starless} variant (left) and results of \ac{ADI} applied to them (right).}
  \label{fig:\plotside:apd\plotId}
\end{figure}
