\chapter{Future work}
\label{chap:future}

% compensate using fit with smearing applied

There are a number of ways to improve the results. We advise to look into \ac{RDI} instead of \ac{ADI}. \ac{RDI} will have far less self subtraction then \ac{ADI} and good results have been achieved imaging disks with \ac{RDI} and coronagraph \cite{rdi} without self-subtraction artifacts. 

For \ac{RDI} an observer collects reference images from stars observed on the same night or from an archive. Then an \ac{PSF} reconstruction algorithm is used on those reference images \cite{rdi}. Finally the reference is subtracted from the target image. This will subtract the speckle pattern while limiting self-subtraction \cite{rdi_2}. A major challenge with \ac{RDI} is creating a good reference \ac{PSF}.

The \ac{ADI} routine might be improved using only a few images as close in time as possible as reference sequence \cite{Marois_2006}. As the instrument \ac{PSF} has evolved less the complete \acp{PSF} are more similar and \ac{ADI} will remove more speckles. If done improperly however this can increase self subtraction. The main source for the artifacts is self subtraction of the disk. When the mean image of the input is subtracted from all the input images pixel values might become negative. These should never be negative exposure. Clipping the \textit{input with mean subtracted} to a minimum value of zero might reduce self subtraction artifacts.

As the disks are heavily deformed after \ac{ADI} trying to fit the result \ac{ADI} result to \textit{starless} disk models with \ac{ADI} applied could help find the parameters of the disk. This could even be extended to training a neural network to generate a disk model given an output of \ac{ADI}.

Finally fitting a general disk model to the observed image would work well in detecting disks however it will be challenging to find undiscovered or detailed features.
