\chapter{Results}

Here we present the results of the paramater study. We have varied the inclination, rotation and number of rings theire sizes. All the combinations presented here are listed in \autoref{tab:disks}. 

For each of these disks we show with increased disk brightness: the disk model with, the end result after ADI and three more images discussed in \autoref{sec:paramstudy}. And we show the disk model at realistic brightness.
\\

\begin{table}
    \begin{tabular}{cccccc}  
    \toprule
    \multicolumn{5}{c}{\thead{Rings}} \\
    \cmidrule(r){3-4}
    \thead{Disk}      & \thead{Figure Number}      & \thead{Number} & \thead{Inner and Outer\\radius} & \thead{Inclination\\ (degrees)} & \thead{Initial Rotation\\ (degrees)}\\
    \midrule
    A         & \ref{fig:A} &   1    & 2 - 4                        & 1           & 30\\
              &             &        & 4 - 5\\
    B         & \ref{fig:B} &   2    & 13.65                                 & 80          & 1\\
    \bottomrule
    \end{tabular}
    \caption{An overview of all the disks we look at. If a disk has multiple rings their inner and outer dimensions are listed on different lines.}
    \label{tab:disks}
\end{table}


%\begin{enumerate}[I]
%\item An exposure of the model with a central star, using a distorted \ac{PSF}.
%\item The result of \ac{ADI} for the model with a central star using distorted \acp{PSF}.

%\item An exposure of the model without a central star, using the undistorted \ac{PSF}.
%\item The result of \ac{ADI} for the model without a central star using the undistorted \ac{PSF}. %TODO explain how we center without star
%\end{enumerate}

\begin{figure}[h!]
  %TODO add end result after ADI at realistic brightness
  \newcommand{\plotnumb}{7}
  \begin{subfigure}[t]{0.6\textwidth}
    \includegraphics[width=\linewidth]{gfx/plots/results/\plotnumb_model}
    \caption{The realistic disk model with star, brightness relative to the star}
  \end{subfigure}% this comment sign needs to be here for the images to be on the same line      
  \begin{subfigure}[t]{0.6\textwidth}
    \includegraphics[width=\linewidth]{gfx/plots/results/\plotnumb_right_ADI_rbri}
    \caption{\ac{ADI} applied to the realistic disk model with star}
  \end{subfigure}
  
  \begin{subfigure}[b]{0.6\textwidth}
    \includegraphics[width=\linewidth]{gfx/plots/results/\plotnumb_right_noReductin}
    \caption{A simulated observation of the brighter model. This models brightness is 100 times larger then the realistic model.}
  \end{subfigure}% this comment sign needs to be here for the images to be on the same line      
  \begin{subfigure}[b]{0.6\textwidth}
    \includegraphics[width=\linewidth]{gfx/plots/results/\plotnumb_right_ADI}
    \caption{\ac{ADI} applied to simulated observation with the brighter disk model where the brightness of the disk relative to the star has been increased by 100 times.}
  \end{subfigure}
  
  \begin{subfigure}[t]{0.6\textwidth}
    \centering
    \includegraphics[width=\linewidth]{gfx/plots/results/\plotnumb_no_star_right_noReductin}
    \caption{Only the light of the disk}
  \end{subfigure}% this comment sign needs to be here for the images to be on the same line
  \begin{subfigure}[t]{0.6\textwidth}
    \centering
    \includegraphics[width=\linewidth]{gfx/plots/results/\plotnumb_no_star_right_ADI}
    \caption{\ac{ADI} applied to only the light coming from the disk}
  \end{subfigure}


  \caption{Results of appying \ac{ADI} to disk model A}
  \label{fig:A}
\end{figure}
