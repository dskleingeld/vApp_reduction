\chapter{Results}

Here we present the results of the paramater study. We have varied the inclination, rotation and number of rings theire sizes. All the combinations presented here are listed in \autoref{tab:disks}. 

For each of these disks we show with increased disk brightness: the disk model with, the end result after ADI and three more images discussed in \autoref{sec:paramstudy}. And we show the disk model at realistic brightness.
\\

\begin{table}
    \begin{tabular}{cccccc}  
    \toprule
    \multicolumn{5}{c}{\thead{Rings}} \\
    \cmidrule(r){3-4}
    \thead{Disk}      & \thead{Figure Numbers}      & \thead{Number} & \thead{Inner and Outer\\radius} & \thead{Inclination\\ (degrees)} & \thead{Initial Rotation\\ (degrees)}\\
    \midrule
    A         & \ref{fig:A} &   1    & 2 - 4                        & 1           & 30\\
              &             &        & 4 - 5\\
    B         & \ref{fig:B} &   2    & 13.65                                 & 80          & 1\\
    \bottomrule
    \end{tabular}
    \caption{An overview of all the disks we look at. If a disk has multiple rings their inner and outer dimensions are listed on different lines.}
    \label{tab:disks}
\end{table}


%\begin{enumerate}[I]
%\item An exposure of the model with a central star, using a distorted \ac{PSF}.
%\item The result of \ac{ADI} for the model with a central star using distorted \acp{PSF}.

%\item An exposure of the model without a central star, using the undistorted \ac{PSF}.
%\item The result of \ac{ADI} for the model without a central star using the undistorted \ac{PSF}. %TODO explain how we center without star
%\end{enumerate}


\begin{figure}[h!]
  %TODO add end result after ADI at realistic brightness
  \newcommand{\plotnumb}{7}
  \begin{subfigure}[t]{0.6\textwidth}
    \includegraphics[width=\linewidth]{gfx/plots/results/\plotnumb_model}
    \caption{The disk model, the brighness is relative to the center star.}
  \end{subfigure}% this comment sign needs to be here for the images to be on the same line      
  \begin{subfigure}[t]{0.6\textwidth}
    \includegraphics[width=\linewidth]{gfx/plots/results/\plotnumb_right_ADI_rbri}
    \caption{\ac{ADI} applied to the disk model, the disk can not be seen only the residuals of the post processing}
  \end{subfigure}
  
  \begin{subfigure}[t]{0.6\textwidth}
    \includegraphics[width=\linewidth]{gfx/plots/results/\plotnumb_right_noReductin}
    \caption{A simulated exposure of the \textit{brighter} variant of the disk model. Due to its unrealistcily high disk brightness we can see where in the \ac{PSF} the disk lies.}
  \end{subfigure}% this comment sign needs to be here for the images to be on the same line      
  \begin{subfigure}[t]{0.6\textwidth}
    \includegraphics[width=\linewidth]{gfx/plots/results/\plotnumb_right_ADI}
    \caption{\ac{ADI} applied simulated exoposure of the \textit{brighter} variant (left). It is hard to discern disk features from \ac{PSF} residuals.}
  \end{subfigure}
  
  \begin{subfigure}[t]{0.6\textwidth}
    \centering
    \includegraphics[width=\linewidth]{gfx/plots/results/\plotnumb_no_star_right_noReductin}
    \caption{A simulated exposure of the \textit{brighter} variant of the disk model. We see where the light of the disk gets spread by the PSF. The disks two rings can clearly be distinguished}
  \end{subfigure}% this comment sign needs to be here for the images to be on the same line
  \begin{subfigure}[t]{0.6\textwidth}
    \centering
    \includegraphics[width=\linewidth]{gfx/plots/results/\plotnumb_no_star_right_ADI}
    \caption{\ac{ADI} applied simulated exoposure of the \textit{starless} variant (left). The outer disk can still be recognised though varius gaps have appeared. The inner disk has become really thin and has two large blobs at opposite side on its semi major axis.}
  \end{subfigure}


  \caption{On the left in decending order: the disk model and generated exposures of its \textit{brighter} and \textit{starless} variants. Then on the right the results of \ac{ADI} applied to the generated exposure sets of the left models.}
  \label{fig:A}
\end{figure}
