\chapter{Results}

Here we present the results of the parameter study. We have varied the inclination, rotation number of rings and the sizes of the ring(s). The combinations presented here are listed in \autoref{tab:disks}. 

As mentioned in \label{sec:paramstudy} we apply \ac{ADI} to the disk model and its \textit{brighter} and \textit{starless} variant. 

The normal disk model could not be recognized in any generated image thus no image of it is shown. 
\\

\begin{table}
    \begin{tabular}{cccccc}  
    \toprule
    \multicolumn{5}{c}{\thead{Rings}} \\
    \cmidrule(r){3-4}
    \thead{Disk}      & \thead{Figure Numbers}      & \thead{Number} & \thead{Inner and Outer\\radius (AU)} & \thead{Inclination\\ (degrees)} & \thead{Initial Rotation\\ (degrees)}\\
    \midrule
    A         & \ref{fig:A} &   1    & 1 - 2                        & 1           & 30\\
              &             &        & 3 - 4\\
    \bottomrule
    \end{tabular}
    \caption{An overview of all the disks we look at. If a disk has multiple rings their inner and outer dimensions are listed on different lines.}
    \label{tab:disks}
\end{table}

\begin{figure}[h!]
  \newcommand{\plotnumb}{0}
  
  %define (\edef) the variable \side to l or r depending if we are at an odd
  %or even page
  \checkoddpage 
  \edef\side{\ifoddpage l\else r\fi}%
  
  \makebox[\textwidth][\side]{%
  
  \begin{minipage}[t]{1.2\textwidth}
      \begin{subfigure}[t]{0.6\textwidth}
        \includegraphics[width=\linewidth]{gfx/plots/results/\plotnumb_model}
        \caption{The disk model, the brightness is relative to the center star.}
      \end{subfigure}% this comment sign needs to be here for the images to be on the same line      
      \begin{subfigure}[t]{0.6\textwidth}
        \includegraphics[width=\linewidth]{gfx/plots/results/\plotnumb_right_ADI_rbri}
        \caption{\ac{ADI} applied to the disk model, the disk can not be seen only the \ac{PSF} residuals}
      \end{subfigure}
      
      \begin{subfigure}[t]{0.6\textwidth}
        \includegraphics[width=\linewidth]{gfx/plots/results/\plotnumb_right_noReductin}
        \caption{A simulated image of the \textit{brighter} variant of the disk model. Due to its unrealistically high disk brightness we can see where in the \ac{PSF} the disk lies.}
      \end{subfigure}% this comment sign needs to be here for the images to be on the same line      
      \begin{subfigure}[t]{0.6\textwidth}
        \includegraphics[width=\linewidth]{gfx/plots/results/\plotnumb_right_ADI}
        \caption{\ac{ADI} applied simulated image of the \textit{brighter} variant (left). It is hard to discern disk features from \ac{PSF} residuals.}
      \end{subfigure}
      
      \begin{subfigure}[t]{0.6\textwidth}
        \centering
        \includegraphics[width=\linewidth]{gfx/plots/results/\plotnumb_no_star_right_noReductin}
        \caption{A simulated image of the \textit{brighter} variant of the disk model. We see where the light of the disk gets spread by the PSF. The disks two rings can clearly be distinguished}
      \end{subfigure}% this comment sign needs to be here for the images to be on the same line
      \begin{subfigure}[t]{0.6\textwidth}
        \centering
        \includegraphics[width=\linewidth]{gfx/plots/results/\plotnumb_no_star_right_ADI}
        \caption{\ac{ADI} applied simulated image of the \textit{starless} variant (left). The outer disk can still be recognized though various gaps have appeared. The inner disk has become really thin and has two large blobs at opposite side on its semi major axis.}
      \end{subfigure}
  \end{minipage}
  }%

  \caption{On the left in descending order: the disk model and generated images of its \textit{brighter} and \textit{starless} variants. Then on the right the results of \ac{ADI} applied to the generated image sets of the left models.}
  \label{fig:A}
\end{figure}
