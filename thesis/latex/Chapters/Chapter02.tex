% Chapter X

\chapter{Disks} % Chapter title

\label{ch:disks} % For referencing the chapter elsewhere, use \autoref{ch:name} 

%----------------------------------------------------------------------------------------

\section{Section Title}

A newly formed star is enveloped in a disk of gass and dust, a protoplanatary disk.

--something about the components/dust/gass ref{williams}

From it planets can form. The process of planat formation is not yet understoot. Direct imaging of newly formed planets is hard. Objects inside a disk disturb the shape and create features in the disk. By observing the features of a disk we can learn more about planet formation.

%------------------------------------------------

\subsection{Catogorisation}

Roughly speaking disks seem to be ring or spiral shaped with some forms in between. /ref{garufi} classifies them into 6 catogories in /fig{sketch garufi}. 

\begin{figure}[h]
    \label{catagories}
    \caption{Sketch summarizing the different classifications of protoplenatary disks proposed by /ref}
    \centering
    \includegraphics[width=8cm]{gfx/catogories}
\end{figure}

They then conclude that: 
-faint disks are young
-spiral disks are almost starting theire main sequence
-ring disk have no outer stellar companion (??)  

%------------------------------------------------

\subsection{Challenging to Obeserve}

It is challanging to observe a disk since its brightness is low compared to the star. To get an upper limit on the brightness of a disk assume all star light that hits the disk is reflected towards us. The light from a star drops quadratically as it gets farther away. Thus we will never have a disk brightness exeeding $1/R^2$. A disk at $1$ AU from an run like star will have a brightness \num{2.1e-5} of the star. At $5$ AU this drops to \num{8.6e-7}. 

%Currently there is only one planet that has been detected with direct imaging within a 1 EU radius, however it orbits an ultracool dwarf /ref{http://exoplanet.eu/catalog/de0823-49_b/}.

%----------------------------------------------------------------------------------------
