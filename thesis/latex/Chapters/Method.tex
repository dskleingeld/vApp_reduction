% Chapter X

\chapter{Data generation} % Chapter title
\label{ch:data_gen} % For referencing the chapter elsewhere, use \autoref{ch:name} 

We choose not to use on sky data but create our own. This presents a number of advantages though the main reason is the lack of disk observations using the vApp. Generating the data from a model allows us to vary paramaters as we wish. We can try a reduction method on a simple slightly inclined disk or a near face on disk with many rings. By reducing images of morpholocally differing disks we can map out how well a reduction method works for each morphology. Further more we can clearly separate artifacts created during data reduction from disk features when we know what the disk looks like.

%------------------------------------------------




\section{Disk model}

We use a 2-dimensional disk model based on the work by \cite{Pieter_Okko}. The model has 4 basic parameters: 

\begin{description}
\item[Inclination] The angle the disk is tilted towards the observer. A 0 degrees inclination gives a face on disk and 90 degrees a horizontal line being an edge on disk. As illustrated in \ref{fig:inclination}
\item[Position angle] After inclination the disk can be rotated around the line of sight from the observer, rotation to the left is positive. See \ref{fig:pos_angle}
\item[Inner and Outer radius] Many disks start and stop at some radius from star. The inner and outer radius are relative to the field size that defaults to 10. An Inner radius of 2 gives a hole in the disk with a diameter 20\% of the image width. 
\end{description}

\begin{figure}[h!]
  \centering
  \begin{subfigure}[b]{0.4\textwidth}
    \includegraphics[width=\textwidth]{gfx/inclination}\label{fig:pos_angle}
    \caption{Inclination angle from: \cite{Pieter_Okko}}

  \end{subfigure}
  \begin{subfigure}[b]{0.4\textwidth}
    \includegraphics[width=\textwidth]{gfx/pos_angle}\label{fig:inclination}
    \caption{Position angle, from: \cite{Pieter_Okko}}
  \end{subfigure}
  \label{fig:1}
\end{figure}

The disk is modeled as optically thick and does not emit light on its own. At a certain radii between the inner and outer radii the disks brightness is given by:

\begin{equation}
B(r) = B_{star} \cdot {\left(\frac{r}{R_{star}}\right)}^2
\end{equation}

We rasterize the image onto a resolution of 200 by 200 pixels. This not only speeds up our calculations it also is the expected output of an actual observing session. The model behaves well for most disks however on this resolution features that are small in the observers plane are pixilated as we see in \autoref{fig:disk} below. This poses no problem as we do not expect to resolve such features.

\begin{figure}[h!]
  \begin{subfigure}[b]{0.5\textwidth}
    \includegraphics[width=1.2\linewidth]{gfx/disk_45_60}
    \caption{model output for an inclination of $60\deg$, an inner radius of $2$ and an outer radius of $5$}
  \end{subfigure}
  \begin{subfigure}[b]{0.5\textwidth}
    \includegraphics[width=1.2\linewidth]{gfx/disk_45_10_22_23}
    \caption{model output for an inclination of $80\deg$, an inner radius of $2.2$ and an outer radius of $2.3$}
  \end{subfigure}
  \caption{two disks created by the model with different size and inclination both with a position angle of 45 degrees}
  \label{fig:disk}
\end{figure}

%------------------------------------------------




\section{Psf generation}

To simulate an observation do not use a single psf as given by /cite{vApp} as during an observation the combined psf of the atmosphere and instrument changes. A complete simulation of these effects is out of the scope of this thesis. We instead decided to try and approach a similar morphology to the on sky psf. 

First we tried modifying a give single vApp psf to get a set of disturbed psfs. To achieve this a pattern is added to the foerier transform of the psf before transforming it back from foerier space, see the equation below. The pattern is then shifted for every timestap.

\begin{equation}
\mathscr{F}_{2d}^{-1} \Big( intensity \cdot \mathscr{F}_{2d}(psf) \cdot pattern + (1-intensity) \cdot \mathscr{F}_{2d}(psf) \Big)
\end{equation}

The best results where achieved using a grid of blurred circles as pattern, see \autoref{fig:patterns}. Note the distortions to the psf are clusterd around the center of its peaks but not randomly spread. Clearly this will not do.

\begin{figure}[h!]
      \begin{subfigure}[t]{0.5\textwidth}
        \includegraphics[width=1.2\linewidth]{gfx/0_old_psf_diff1}
        \caption{difference between two psfs created using this method, the pattern has shifted 10 pixels}
      \end{subfigure}% this comment sign needs to be here for the images to be on the same line
      \begin{subfigure}[t]{0.5\textwidth}
        \includegraphics[width=1.2\linewidth]{gfx/0_old_psf_diff2}
        \caption{difference between two psfs created using this method, the pattern has shifted 5 pixels}
      \end{subfigure}
      
      \begin{subfigure}[]{0.5\textwidth}
        \includegraphics[width=1.2\linewidth]{gfx/0_pattern}
        \caption{pattern that is shifted and applied to the foerier space of a psf}
      \end{subfigure}
      
  \caption{Pattern (a) and the diffrences between psfs **distorted** with this pattern using the method described above}
  \label{fig:patterns}
\end{figure}

Then we tried a simulation of the psfs using HCIPy, an open-source object-oriented framework written in Python for performing end-to-end simulations of high-contrast imaging instruments \cite{hcipy}. The framework is used to do a very rough simulation. We use the provided methods to create a multi-layer atmospheric model that changes in time. Then we use that model, the vApp amplitude and vApp phase screen to generate a series of psfs. Instead of modelling an active optics system we modified the fried parameter for the atmosphere to get similar morphological changes in time to the availible on sky images. This is sufficient as the disk will always lie within the .....radius of correction???..... which means the limiting effect of an AO system on the disk will only be a increase in resolution by reducing the seeing effect.

Wu use a telescope diamater of $8.2 meters$ and a wavelength, of $1\cdot 10^{-6} meters$ for the simulation. We got the right morphology with a fried paramater of 4m. To put this into perspective: to simulate excellent seeing conditions we would use 20cm. With these setting we use hcipy to create an orderd set of psfs changing through time as the simulated atmosphere evolves. See \autoref{fig:psfs_evolving} for the diffrence between the first psf in a set and later psfs in the same set. Note that as time evolves the differences grow. 

%https://tex.stackexchange.com/questions/169219/subcaption-third-figure-breaks-column
%https://tex.stackexchange.com/questions/139214/how-to-make-a-subfigure-span-the-two-columns-in-ieeetrans-style
\begin{figure}[h!]
  
      \begin{subfigure}[b]{0.5\textwidth}
        \includegraphics[width=1.2\linewidth]{gfx/0_psf_diff_007png}
        \caption{simulated psf: 1e - 2e, 0.7 seconds difference}
      \end{subfigure}% this comment sign needs to be here for the images to be on the same line
      \begin{subfigure}[b]{0.5\textwidth}
        \includegraphics[width=1.2\linewidth]{gfx/0_on_sky_psf_diff_007}
        \caption{on sky psf: 1th - 2th, *** seconds difference}
      \end{subfigure}      
      
      \begin{subfigure}[b]{0.5\textwidth}
        \includegraphics[width=1.2\linewidth]{gfx/0_psf_diff_070png}
        \caption{simulated psf: 1th - 10th, 7 seconds difference}
      \end{subfigure}% this comment sign needs to be here for the images to be on the same line
      \begin{subfigure}[b]{0.5\textwidth}
        \includegraphics[width=1.2\linewidth]{gfx/0_on_sky_psf_diff_070}
        \caption{on sky psf: 1th - 10th, *** seconds difference}
      \end{subfigure}
            
      \begin{subfigure}[b]{0.5\textwidth}
        \includegraphics[width=1.2\linewidth]{gfx/0_psf_diff_140png}
        \caption{simulated psf: 1th - 50th, 14 seconds difference}
      \end{subfigure}% this comment sign needs to be here for the images to be on the same line
      \begin{subfigure}[b]{0.5\textwidth}
        \includegraphics[width=1.2\linewidth]{gfx/0_on_sky_psf_diff_140}
        \caption{on sky psf: 1th - 50th, *** seconds difference}
      \end{subfigure}

  %\end{minipage}\quad%

  \caption{Differences between the first psf of an observation and later psfs. Compared between simulated observations on the left and on sky data on the right. The psfs where aligned then normalised on the maximum of the leakage term, finally the absolute value of the diffrence between the first and n-th psf was taken. The second, 10th and 50th simulated psf are 0.7, 7 and 14 seconds apart in the simulation}
  \label{fig:psfs_evolving}
\end{figure}

%%%%%%%%%%%%%TODO radius of correction noemen

%----------------------------------------------------------------------------------------





\section{Combinging}

To get a dataset that simulates an obeservation we create an orderd set of disk images each one rotated a bit to eachother. This accounts for the field rotation caused by observing with an alt azimuth telescope. Now to get the simulated observation an orderd set is created by convolving each n-th image from the disk set with the n-th image of the psf set. These are the simulated observations through time. 

Using variations on this method we create 3 different sets.

\begin{enumerate}
\item Place a single pixel with value one in the center of the disk before going through the above procedure. This is our observation set.
\item Swap the disk with the single pixel in the center representing the star. This set we use to check what artefacts are created by the reduction 
\item Take the convolution of the n-th disk image with the first psf image of the psf set. This way we lose all atmospheric effects, since we havent placed a star in the center we see only te effect of the vApp on the disk image. 
\end{enumerate}
